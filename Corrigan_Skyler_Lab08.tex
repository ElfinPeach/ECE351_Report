%%%%%%%%%%%%%%%%%%%%%%%%%%%%%%%%%%%%%%%%%%%%%%%%%%%%%%%%%%%%%%%%%%%%%
%                                                                   %
% Skyler Corrigan                                                   %
% ECE 351-52                                                        %
% Lab 08                                                            %
% 10/13/2022                                                        %
% GitHub Reports: https://github.com/ElfinPeach/ECE351_Report.git   %
% GitHub Codes: https://github.com/ElfinPeach/ECE351_Code.git       %
%                                                                   %
%%%%%%%%%%%%%%%%%%%%%%%%%%%%%%%%%%%%%%%%%%%%%%%%%%%%%%%%%%%%%%%%%%%%%

%%% DOCUMENT PREAMBLE %%%

\documentclass[12pt,a4paper]{article}

\usepackage{listings}

\usepackage[utf8]{inputenc}
\usepackage[greek,english]{babel}
\usepackage{alphabeta} 

\usepackage[pdftex]{graphicx}
\usepackage[top=1in, bottom=1in, left=1in, right=1in]{geometry}

\linespread{1.06}
\setlength{\parskip}{8pt plus2pt minus2pt}

\widowpenalty 10000
\clubpenalty 10000

\newcommand{\eat}[1]{}
\newcommand{\HRule}{\rule{\linewidth}{0.5mm}}

\usepackage[official]{eurosym}
\usepackage{enumitem}
\setlist{nolistsep,noitemsep}
\usepackage[hidelinks]{hyperref}

\begin{document}

%===========================================================
\begin{titlepage}
\begin{center}

% Top 
%\includegraphics[width=0.55\textwidth]{cut-logo-en}~\\[2cm]


% Title
\HRule \\[0.4cm]
{ \LARGE 
  \textbf{Project Report for ECE 351}\\[0.4cm]
  \emph{Lab 08 - Fourier Series Approximation of a Square Wave}\\[0.4cm]
}
\HRule \\[1.5cm]



% Author
{ \large
  Skyler Corrigan \\[0.1cm]
}

\vfill

%\textsc{\Large Cyprus University of Technology}\\[0.4cm]\textsc{\large Department of Electrical Engineering,\\Computer Engineering \& Informatics}\\[0.4cm]


% Bottom
{\large \today}

% Links
{ \large
ECE351 Code Repository: 
\hyperlink{$https://github.com/ElfinPeach/ECE351_Code.git$}{$https://github.com/ElfinPeach/ECE351_Code.git$}

ECE351 Report Repository: 
\hyperlink{$https://github.com/ElfinPeach/ECE351_Report.git$}{$https://github.com/ElfinPeach/ECE351_Report.git$}
}
 
\end{center}
\end{titlepage}

%\begin{abstract}
%\lipsum[1-2]
%\addtocontents{toc}{\protect\thispagestyle{empty}}
%\end{abstract}

\newpage

%===========================================================
\tableofcontents
\addtocontents{toc}{\protect\thispagestyle{empty}}
\newpage
\setcounter{page}{1}

%===========================================================
%===========================================================
\section{Objective}
This lab is about trying to find the Fourier Series of the following Square Wave.\\
\\
\textit{Figure 1}
\\
\includegraphics[width=6in]{Prelab.png}
\\
\section{Equations}
The following equations can be used to define the wave:
\begin{center}
    $x(t)=\frac{1}{2}a_0+\sum_{n=1}^{\infty}a_k\cos{(k\omega_0t)}$\\
    $a_k=\frac{2}{T}\int_{0}^{T}x(t)\cos{(k\omega_0t)}dt$\\
    $b_k=\frac{2}{T}\int_{0}^{T}x(t)\sin{(k\omega_0t)}dt$\\
    $\omega_0=\frac{2\pi}{T}$
\end{center}
Since this is an odd function, $a_0=a_k=0$.\\
For $b_k$:
\begin{center}
    $b_k=\frac{2}{k\pi}(-cos(k\pi)+1)$\\
    if k is even, $b_k=0$\\
    if k is odd, $b_k=\frac{4}{k\pi}$\\
    Therefore: $x(t)=\sum_{k=1}^{\infty}\frac{4}{k\pi}\sin{(k\omega_0t)}$ for when k is even.
\end{center}
Though, for the code, the original $b_k$ function will be used in $x_t$ to simplify the code (plus the code kept breaking when I tried to do it that way...).\\
\section{Code}
The following is the code used in the lab.
\textit{Lab Code}
\begin{lstlisting}[language=Python]
import numpy as np
import matplotlib.pyplot as plt

#---------PT 1-------------------------------------------#
def b_k(n):
    b = (2 / (n * np.pi)) * (1 - np.cos((n) * np.pi))
    return b


def W(period):
    return ((2*np.pi)/period)


def xFourier(t, period, n):
    x_t = 0 #initialization
    for i in np.arange(1, n+1): 
        x_t += (b_k(i))*(np.sin(i * W(period) * t))
    return x_t

#Since this is an odd function, a0 = an = 0

#step size
steps = 1e-1

# t for part 1
start = 0
stop = 20
# Define a range of t. Start at 0 and go to 20 (+a step) 
t = np.arange(start, stop + steps, steps)

#Make arrays to plot against t
x_1 = xFourier(t, 8, 1)
x_3 = xFourier(t, 8, 3)
x_15 = xFourier(t, 8, 15)
x_50 = xFourier(t, 8, 50)
x_150 = xFourier(t, 8, 150)
x_1500 = xFourier(t, 8, 1500)

#plot stuff
plt.figure(figsize=(10, 12))
#space out subplots
plt.subplots_adjust(left=0.1, bottom=0.1, right=0.9,
                    top=1.0, wspace=0.4, hspace=0.4)

#PLOT EVERYTHING!!!!
plt.subplot(6, 1, 1)
plt.plot(t, x_1)
plt.title("Estimation when N=1")
plt.ylabel("x(t) Output")
plt.grid()

plt.subplot(6, 1, 2)
plt.plot(t, x_3)
plt.title("Estimation when N=3")
plt.ylabel("x(t) Output")
plt.grid()

plt.subplot(6, 1, 3)
plt.plot(t, x_15)
plt.title("Estimation when N=15")
plt.ylabel("x(t) Output")
plt.grid()

plt.subplot(6, 1, 4)
plt.plot(t, x_50)
plt.title("Estimation when N=50")
plt.ylabel("x(t) Output")
plt.grid()

plt.subplot(6, 1, 5)
plt.plot(t, x_150)
plt.title("Estimation when N=150")
plt.ylabel("x(t) Output")
plt.grid()

plt.subplot(6, 1, 6)
plt.plot(t, x_1500)
plt.title("Estimation when N=1500")
plt.ylabel("x(t) Output")
plt.grid()

plt.show()

print("a0 and an values for a are zero. This is because the wave is odd and has no offset.")
print("")
print("b1 = ", b_k(1))
print("")
print("b2 = ", b_k(2))
print("")
print("b3 = ", b_k(3))
\end{lstlisting}
\newpage
\section{Results}
The following graphs are the Fourier Series Approximation of the square wave seen in \textit{Figure 1}, iterated at $N={1,3,15,50,150,1500}$.
\\
\textit{Figure 2}
\\
\includegraphics[width=6in]{Plots.png}
\newpage
\section{Questions}
\subsection{Question 1}
Is x(t) an even or an odd function? Explain why.\\
\\
x(t) is an odd function because $x(t)=-x(-t)$\\
\subsection{Question 2}
Based on your results from Task 1, what do you expect the values of a2, a3, . . . , an to be? Why?\\
\\
I expect all those value to be zero because it's an odd function!
\subsection{Question 3}
How does the approximation of the square wave change as the value of N increases? In what way does the Fourier series struggle to approximate the square wave?\\
\\
The approximation gets better as N increases. Fourier stuggles to approximate the square wave because its lines are straight!
\subsection{Question 4}
What is occurring mathematically in the Fourier series summation as the value of N increases?\\
\\
The Fourier Series is approaching a limit at some time value.\\
\subsection{Question 5}
Leave any feedback on the clarity of lab tasks, expectations, and deliverables.\\
\\
It may be a good idea to, in the prelab anyway, throw out a hint, something like: "Hint: Look up the difference between odd and even functions when approximating a graph." I personally remembered there was a "cheat" to make life easier, but not everyone does.
\end{document}