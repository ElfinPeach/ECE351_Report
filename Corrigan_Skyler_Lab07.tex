%%%%%%%%%%%%%%%%%%%%%%%%%%%%%%%%%%%%%%%%%%%%%%%%%%%%%%%%%%%%%%%%%%%%%
%                                                                   %
% Skyler Corrigan                                                   %
% ECE 351-52                                                        %
% Lab 04                                                            %
% 09/22/2022                                                        %
% GitHub Reports: https://github.com/ElfinPeach/ECE351_Report.git   %
% GitHub Codes: https://github.com/ElfinPeach/ECE351_Code.git       %
%                                                                   %
%%%%%%%%%%%%%%%%%%%%%%%%%%%%%%%%%%%%%%%%%%%%%%%%%%%%%%%%%%%%%%%%%%%%%

%%% DOCUMENT PREAMBLE %%%

\documentclass[12pt,a4paper]{article}

\usepackage{listings}

\usepackage[utf8]{inputenc}
\usepackage[greek,english]{babel}
\usepackage{alphabeta} 

\usepackage[pdftex]{graphicx}
\usepackage[top=1in, bottom=1in, left=1in, right=1in]{geometry}

\linespread{1.06}
\setlength{\parskip}{8pt plus2pt minus2pt}

\widowpenalty 10000
\clubpenalty 10000

\newcommand{\eat}[1]{}
\newcommand{\HRule}{\rule{\linewidth}{0.5mm}}

\usepackage[official]{eurosym}
\usepackage{enumitem}
\setlist{nolistsep,noitemsep}
\usepackage[hidelinks]{hyperref}


\begin{document}

%===========================================================
\begin{titlepage}
\begin{center}

% Top 
%\includegraphics[width=0.55\textwidth]{cut-logo-en}~\\[2cm]


% Title
\HRule \\[0.4cm]
{ \LARGE 
  \textbf{Project Report for ECE 351}\\[0.4cm]
  \emph{Lab 07 - Block Diagrams and System Stability}\\[0.4cm]
}
\HRule \\[1.5cm]



% Author
{ \large
  Skyler Corrigan \\[0.1cm]
}

\vfill

%\textsc{\Large Cyprus University of Technology}\\[0.4cm]\textsc{\large Department of Electrical Engineering,\\Computer Engineering \& Informatics}\\[0.4cm]


% Bottom
{\large \today}

% Links
{ \large
ECE351 Code Repository: 
\hyperlink{$https://github.com/ElfinPeach/ECE351_Code.git$}{$https://github.com/ElfinPeach/ECE351_Code.git$}

ECE351 Report Repository: 
\hyperlink{$https://github.com/ElfinPeach/ECE351_Report.git$}{$https://github.com/ElfinPeach/ECE351_Report.git$}
}
 
\end{center}
\end{titlepage}

%\begin{abstract}
%\lipsum[1-2]
%\addtocontents{toc}{\protect\thispagestyle{empty}}
%\end{abstract}

\newpage

%===========================================================
\tableofcontents
\addtocontents{toc}{\protect\thispagestyle{empty}}
\newpage
\setcounter{page}{1}

%===========================================================
%===========================================================
\section{Objective}
This lab is designed to work on block diagrams and Laplace domains. The block diagram used in this lab is shown below.\\
\\
\textit{Figure 1}
\\
\includegraphics[width=6in]{Block Diagram.png}
\\
\section{Equations}
The following equations are used in throughout the lab report, and will be referenced by their Roman Numeral numbers.
\begin{center}
    $G(s)=\frac{s+9}{(s^2-6s-16)(s+4)}=\frac{5}{24(s+4)}-\frac{7}{20(s+2)}+\frac{17}{120(s-8)}$ (I)\\
    $A(s)=\frac{s+4}{s^2+4s+3}=\frac{3}{2(s+1)}-\frac{1}{2(s+3)}$ (II)\\
    $B(s)=s^2+26s+168$ (III)\\
    $H_{open}(s)= \frac{s+9}{s^4-2s^3-37s^2-82s-48}$ (IV)\\
    $H_{closed}(s)=\frac{1+13s+36}{2s^5+41s^4+500s^3+2995s^2+6878s+4344}$ (V)
\end{center}
\section{Code}
The following is the entirety of the code I used for this lab.\\
\\
\textit{Lab Code}
\begin{lstlisting}[language=Python]
import numpy as np
import matplotlib.pyplot as plt
import scipy.signal as sig

        #Universal Stuff for Lab

    #Step size
steps = 1e-2

    #t for part 1
start = 0
stop = 10
    #Define a range of t. Start at 0 and go to 20 (+a step)
t = np.arange(start, stop + steps, steps)

#--------------PART 1, Open Loop Test------------------------------------#
    #Find the roots and the poles
    
    #A(s)=(s+4)/((s^2+4s+3))
A_num = [1, 4]
A_den = [1, 4, 3]

    #B(s)=s^2+26s+168
B_num = [1, 26, 168]
B_den = [1]

    #G(s)=(s+9)/((s^2+-6s-16)(s+4))=(s+4)/(s^3-2s^2-40s-64)
G_num = [1, 4]
G_den = [1, -2, -40, -64]

Azeros, Apoles, Again = sig.tf2zpk(A_num, A_den)
Bzeros, Bpoles, Bgain = sig.tf2zpk(B_num, B_den)
Gzeros, Gpoles, Ggain = sig.tf2zpk(G_num, G_den)

print("Zeros for A: ", Azeros)
print("")
print("Poles for A: ", Apoles)
print("")
print("Zeros for B: ", Bzeros)
print("")
print("Poles for B: ", Bpoles)
print("")
print("Zeros for G: ", Gzeros)
print("")
print("Poles for G: ", Gpoles)
print("")

H_open_num = [1, 9]
H_open_den = [1, -2, -37, -82, -48]

H_open_zeros, H_open_poles, H_open_gain
sig.tf2zpk(H_open_num, H_open_den)
print("Zeros of the Open Loop H(s)")
print(H_open_zeros)
print("")
print("Poles of the Open loop H(s)")
print(H_open_poles)
print("")

stepTOpen, stepHOpen = sig.step((H_open_num, H_open_den), 
T = t)

plt.figure(figsize=(10,7))
plt.plot(stepTOpen, stepHOpen)
plt.grid()
plt.xlabel("Time")
plt.ylabel("Output")
plt.title("Open Loop H(s) Step Response")

#-------------PART 2, Closed Loop Test---------------------------------------#

    #Finding the Numerator and Denomenator of transfer function
H_closed_num = sig.convolve([1,4],[1,9])

part_den = sig.convolve([1,1],[1,3])
H_closed_den = sig.convolve(part_den,[2,33,362,1448])

print("Closed loop numerator = ", H_closed_num)
print("Closed loop denemonator = ", H_closed_den)
print()

    #Numbers I got by hand:
#H_clo_num = [1, 13, 36]
#H_clo_den = [2, 41, 500, 2995, 6878, 4344]

H_closed_zeros, H_closed_poles, H_closed_gain = 
sig.tf2zpk(H_closed_num, H_closed_den)

print("Poles of the Closed loop H(s):")
print(H_closed_poles)
print("")

    #Define transfer function!
stepTClosed, stepHClosed = sig.step((H_closed_num, 
H_closed_den), T = t)

    #Make plots for pt1
plt.figure(figsize=(10,7))
plt.plot(stepTClosed, stepHClosed)
plt.grid()
plt.xlabel("Time")
plt.ylabel("Output")
plt.title("Closed Loop H(s) Step Response")

#--------SHOW PLOTS--------------------#
plt.show()
\end{lstlisting}
\newpage
\section{Part 1}
Running the code gives the following values for poles and zeros for A, B, and G.
\begin{center}
\begin{tabular}{ |c|c|c| } 
 \hline
 Equation & Holes & Poles \\ 
 \hline \hline
 A(s) & -4 & -3, -1 \\ 
 \hline
 B(s) & -14, -12 & N/A \\ 
 \hline
 G(s) & -4 & 8, -4, -2 \\ 
 \hline
\end{tabular}
\end{center}
These values are consist ant with Equations (I), (II), and (III). \\
\\
The Open Loop equation for the system found in \textit{Figure 1} can be found in (IV). Since the limit as "s" goes to 0 is negative, the system is unstable.\\
\\
The following figure shows the output for the Open Loop Step Response of (IV).\\
\\
\textit{Figure 2: Open Loop Step Response}
\\
\includegraphics[width=6in]{Open Loop.png}
\\
This shows an unstable system, as the values shoot off to infinity as time increases.
\newpage
\section{Part 2}
The Closed Loop equation for the found in \textit{Figure 1} can be found in (V). Since the limit as "s" goes to 0 is positive, the system is stable.\\
\\
The following figure shows the output for the Open Loop Step Response of (V).\\
\\
\textit{Figure 3: Closed Loop Step Response}
\\
\includegraphics[width=6in]{Closed Loop.png}
\\
This shows a stable system because the system plateaus (reaches a limit) as time progresses.
\end{document}