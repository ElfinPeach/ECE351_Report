%%%%%%%%%%%%%%%%%%%%%%%%%%%%%%%%%%%%%%%%%%%%%%%%%%%%%%%%%%%%%%%%%%%%%
%                                                                   %
% Skyler Corrigan                                                   %
% ECE 351-52                                                        %
% Lab 04                                                            %
% 09/22/2022                                                        %
% GitHub Reports: https://github.com/ElfinPeach/ECE351_Report.git   %
% GitHub Codes: https://github.com/ElfinPeach/ECE351_Code.git       %
%                                                                   %
%%%%%%%%%%%%%%%%%%%%%%%%%%%%%%%%%%%%%%%%%%%%%%%%%%%%%%%%%%%%%%%%%%%%%

%%% DOCUMENT PREAMBLE %%%

\documentclass[12pt,a4paper]{article}

\usepackage{listings}

\usepackage[utf8]{inputenc}
\usepackage[greek,english]{babel}
\usepackage{alphabeta} 

\usepackage[pdftex]{graphicx}
\usepackage[top=1in, bottom=1in, left=1in, right=1in]{geometry}

\linespread{1.06}
\setlength{\parskip}{8pt plus2pt minus2pt}

\widowpenalty 10000
\clubpenalty 10000

\newcommand{\eat}[1]{}
\newcommand{\HRule}{\rule{\linewidth}{0.5mm}}

\usepackage[official]{eurosym}
\usepackage{enumitem}
\setlist{nolistsep,noitemsep}
\usepackage[hidelinks]{hyperref}


\begin{document}

%===========================================================
\begin{titlepage}
\begin{center}

% Top 
%\includegraphics[width=0.55\textwidth]{cut-logo-en}~\\[2cm]


% Title
\HRule \\[0.4cm]
{ \LARGE 
  \textbf{Project Report for ECE 351}\\[0.4cm]
  \emph{Lab 06 - Partial Fraction Expansion}\\[0.4cm]
}
\HRule \\[1.5cm]



% Author
{ \large
  Skyler Corrigan \\[0.1cm]
}

\vfill

%\textsc{\Large Cyprus University of Technology}\\[0.4cm]\textsc{\large Department of Electrical Engineering,\\Computer Engineering \& Informatics}\\[0.4cm]


% Bottom
{\large \today}

% Links
{ \large
ECE351 Code Repository: 
\hyperlink{$https://github.com/ElfinPeach/ECE351_Code.git$}{$https://github.com/ElfinPeach/ECE351_Code.git$}

ECE351 Report Repository: 
\hyperlink{$https://github.com/ElfinPeach/ECE351_Report.git$}{$https://github.com/ElfinPeach/ECE351_Report.git$}
}
 
\end{center}
\end{titlepage}

%\begin{abstract}
%\lipsum[1-2]
%\addtocontents{toc}{\protect\thispagestyle{empty}}
%\end{abstract}

\newpage



%===========================================================
\tableofcontents
\addtocontents{toc}{\protect\thispagestyle{empty}}
\newpage
\setcounter{page}{1}

%===========================================================
%===========================================================
\section{Objective}
This lab is designed to demonstrate partial fraction expansion.\\

\section{Part 1}
The equation that is analyzed is as follows:
\begin{center}
    $y''(t) + 10y'(t) + 24y(t) = x''(t) + 6x'(t) + 12x(t)$
\end{center}
The transfer function that was hand calculated is as follows:
\begin{center}
    $H(s)=\frac{Y(s)}{X(s)}=\frac{s^2+6s+12}{s^2+10s+24}$
\end{center}
Using the Python function to find the roots and poles, they are as follows:\\
\begin{center}
    Roots = [0.5, -0.5, 1]\\
    Poles = [0, -4, -6]\\
\end{center}
These roots and poles show that the Python function is equivalent to the hand defined one.\\
\\
The hand defined and computer defined functions plot the graphs below.
\newpage
\textit{Figure 1}
\\
\includegraphics[width=6in]{Plot 1.png}
\newpage
\section{Part 2}
The function evaluated in this section is as follows:
\begin{center}
    $y^{(5)} + 18y^{(4)} + 218y'''(t) + 2036y''(t) + 9085y'(t) + 25250y(t) = 25250x(t)$
\end{center}
Using the scipy.signal.residue() function, the roots, poles, and K value are as follows:
\begin{center}
    Roots = [1+0.j, -0.486+0.728j, -0.486-0.728j, -0.215+0.j,  0.0929-0.047j  0.093+0.0477j]\\
    Poles = [0.+0.j, -3.+4.j, -3.-4.j, -10.+0.j, -1.+10.j, -1.-10.j]\\
    K = 0.104 $\angle 0.474$ 
\end{center}
These were used to create a plot using the Cosine method, which is:\\
\begin{center}
    $y(t)=\vert k \vert * e^{\alpha t} * \cos{(\omega * t + \angle k)}$
\end{center}
This was compared to the build in step function, and both were plotted as seen below:\\
\
\textit{Figure 2}
\\
\includegraphics[width=6in]{Plot 2.png}
\\
Since both of these plots are equivalent, it shows that both methods result in the same outcome.
\newpage
\section{Appendix}
The following is the code used for the entirety of the lab.\\
\textit{Lab Code}
\begin{lstlisting}[language=Python]
import numpy as np
import matplotlib.pyplot as plt
import scipy.signal as sig

#------------FUNCTIONS----------------------------------------#

#Make a step function using an array t, stepTime, and stepHeight
def stepFunc(t, startTime, stepHeight):
    y = np.zeros(t.shape)
    for i in range(len(t)):
        if(t[i] >= startTime):
            y[i] = stepHeight
    return y

#----------------END FUNCTIONS-----------------------------#

#---------Part 1-------------------------------------------#
    #Define step size
steps = 1e-2

    #t for part 1
start = 0
stop = 2
    #Define a range of t. Start at 0 and go to 20 (+a step)
t = np.arange(start, stop + steps, steps)

    #Prelab stuff
h = ((np.exp(-6 * t)) + (-0.5 * np.exp(-4 * t))+ 0.5) * stepFunc(t, 0, 1)

    #Make the H(s) using the sig.step()
    
num = [1, 6, 12] #Creates a matrix for the numerator
den = [1, 10, 24] #Creates a matrix for the denominator

tout , yStep = sig.step((num , den), T = t)

den_residue = [1, 10, 24, 0]

    #Make and print the partial fraction decomp
roots, poles, _ = sig.residue(num, den_residue)

print("Part 1")
print("Roots =", roots)
print("Poles =", poles)


    #Make plots for pt1
plt.figure(figsize=(10,7))
plt.subplot(2,1,1)
plt.plot(t,h, 'purple')
plt.grid()
plt.ylabel('Hand Solved Output')
plt.title('Plots of Hans Solved and Computer Solved Outputs')

plt.subplot(2,1,2)
plt.plot(t,yStep, 'red')
plt.grid()
plt.ylabel('sig.step Output')


#------------PART 2-------------------------

    #Define step size
steps = 1e-2

    #t for part 1
start = 0
stop = 4.5
    #Define a range of t2. Start at 0 and go to 20 (+a step)
t2 = np.arange(start, stop + steps, steps)

    #Make numerator and denomentaor for sig.residue()
num2 = [25250]
den2 = [1, 18, 218, 2036, 9085, 25250, 0]

R, P, K = sig.residue(num2, den2)

print("")
print("Part 2")
print("Roots =", R)
print("Poles =", P)

    #cosine vethod
yt = 0

    #Range iterates through each root
for i in range(len(R)):
    angleK = np.angle(R[i])
    magK = np.abs(R[i])
    W = np.imag(P[i])
    a = np.real(P[i])
        
    yt += magK * np.exp(a * t2) * np.cos(W * t2 + angleK) * 
        stepFunc(t2, 0, 1)
    
print("K value =", magK)
print("K angle =", angleK)
    
#Make the lib generated step response
den2_step = [1, 18, 218, 2036, 9085, 25250]
tStep2, yStep2 = sig.step((num2,den2_step), T = t2)

    #Show Plots
plt.figure(figsize=(10,7))
plt.subplot(2,1,1)
plt.plot(t2, yt,'purple')
plt.grid()
plt.xlabel('t')
plt.ylabel('y(t)')
plt.title('Cosine Method vs. Lib sig.step Method')


plt.subplot(2,1,2)
plt.plot(tStep2, yStep2, 'red')
plt.grid()
plt.xlabel('t')
plt.ylabel('sig.step y(t)')

#-------------SHOW ALL PLOTS-----------------
plt.show()
\end{lstlisting}
\end{document}